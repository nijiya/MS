% INTRODUCTION
\chapter{Introduction}
\label{chapter:introduction}

The problem of finding a path for an autonomous agent from an initial location to a destination location is a popular problem in real-world applications including robotics, virtual simulations or computer games and has been studied for many years. Thus, many solutions are proposed in computer science society for this issue. Proposed path planning algorithms can be classified into four categories: off-line algorithms \cite{Dijkstra:1959} \cite{AStarHart:1968}, on-line algorithms \cite{RTAStarKorf:1990}, incremental algorithms \cite{DStar:1994}, \cite{Koenig:2002}, \cite{FocussedDStarStentz:1995} and soft computing algorithms \cite{Tarapata:2007}, \cite{Pangilinan}. Off-line path planning algorithms try to find the whole solution before starting the execution, whereas on-line search algorithms require the planning and execution phases to be coupled, such that the agent repeatedly plans and executes the next movement. In dynamic or partially known environments, off-line path planning algorithms suffer from execution time, whereas on-line algorithms yield low quality solutions in terms of path length. Incremental heuristic search algorithms try to merge advantages of both approaches to obtain better execution time without sacrificing optimality. They reuse the information gained from previous iterations and improve it instead of calculating from scratch like off-line search methods. Soft computing algorithms generally come up with evolutionary solutions. Their main perspective is to evaluate and evolve solution quality by time. 

Existing incremental algorithms for path planning problem attempt to minimize path length. However, in many real-world problem domains we see that there are several objectives to be optimized concerning the solution (path) quality. Consider the navigation of an unmanned vehicle from one coordinate to another on a 3D terrain in a warfare setting. The navigation task is defined to be finding a path which is shortest but also the safest among all possibilities considering the existence of opponent forces in a partially known environment due to limited sensor capabilities. Note that shortest path may not be the safest one, on the contrary it might be the most dangerous one. And also the safest path may be the longest one which is unacceptable due to fuel consumption or time thresholds. 

There is a need to generalize the notion of quality of a path to meet specific requirements of complex application domains where several objectives (criteria) that cannot be transformed to each other exist. For example, in our unmanned vehicle example, it is not possible to transform the distance metric to the safety metric, and vice versa. This requirement raises the problem of handling decision making of multiple criteria at the same time. In this study; an incremental path finding algorithm called Multi-Objective D* Lite (MOD* Lite) which extends an existing incremental algorithm, Dynamic A* Lite (D* Lite) \cite{Koenig:2002}, is introduced. MOD* Lite \cite{Oral:2012} can be used in the design of an autonomous mobile agent facing with the problem of navigation in a partially known environment that needs to optimize a predefined set of independent objectives (criteria). The agent might have limited sensor capability and hence partially observe the environment, and furthermore need to optimize multiple objectives at the same time.

In order to show that MOD* Lite generates the optimal and sub-optimal solutions, also a new multi objective genetic path planning (MOGPP) algorithm is designed. This algorithm finds initial paths randomly and its population (solutions) evolve according to a fitness function. MOD* Lite is both compared against MOGPP and MOA* algorithm \cite{MOAStewart:1991}, an offline algorithm, on some test environments that are fully observable. The performance of MOD* Lite is also tested on several partially observable environments guaranteeing the optimal solutions but outperforming the MOA* and MOGPP versions modified for unknown environments.

The following chapters in this thesis are organized as follows: Chapter 2 gives the background and related work for this study. As MOA* is used in experimental studies and D* Lite is used as a base of proposed solution, these algorithms are also detailed in this chapter. The problem definition, characteristics of the environment and proposed solutions (MOD* Lite and MOGPP) are presented in Chapter 3. Experimental studies and their results are stated in Chapter 4. Finally, the conclusion and future studies are provided in Chapter 5.
